% Preamble
\documentclass[10pt,letterpaper]{article}
% Packages
\usepackage[spanish]{babel}
\selectlanguage{spanish}
\usepackage[utf8]{inputenc} % For spanish (and international) letters like acents.
\usepackage{hyperref} % To create hyperlinks within the document.
\usepackage{graphicx} % To include graphics (pictures, images)
\usepackage{verbatim} % For long comments
%\usepackage{color} % For color in text, background text and page background.
\usepackage[usenames,dvipsnames]{xcolor} % For color in text, background text and page background.

% Document environment
\begin{document}

% Top matter
\title{Recomendaciones para el Manual de Catalogación del Acervo de Documental en Video}
\author{Rodrigo Eduardo Colín Rivera, Sergio Amaro Rosas}
\date{\today}
\maketitle

\setcounter{secnumdepth}{0} % Non-numbering sections
\setcounter{tocdepth}{0} % Non-numnbering table of contents

\section{Antecedentes}
El Manual de Catalogación del Acervo de Documental en Video para la Serie de Documentales en Video del Laboratorio Audiovisual de Investigación Social del Instituto Mora es un manual que expone las características de los diversos rubros que integra la ficha de documentación gestada con base en la adaptación de la norma ISAD(G) (\textit{General International Standard Archival Description}).

Durante el desarrollo del sistema de apoyo a la catalogación de archivos audiovisuales se tomó la decisión de crear una \textbf{base de datos} que almacene cada rubro estipulado en dicho manual. Sin embargo, dado que la información debe ser comprensible tanto para personas como para la máquina, es decir, se requieren algunos convenciones para que la información pueda ser procesada de manera automática sin perder la expresividad descrita por las personas o catalogadores.

A continuación se describen los formatos deseables de algunos campos que facilitarán el proceso de automatizar y manejar la información del ccervo de documental en video.

\section{Fechas}

\subsection{Fecha}
Comúnmente este rubro consigna el \textbf{año} en el que se terminó la producción de la obra. La fecha puede comprender un periodo de tiempo determinado indicando el año de inicio y el año de finalización de la obra y en caso de desconocer este dato, se aproxima la fecha entre corchetes. 

Ejemplos de formato recomendado:
{\color{Blue}
\begin{itemize}
	\item \texttt{1956}
	\item \texttt{1948-1949}
	\item \texttt{[1982]}
	\item \texttt{[1992-1993]}
\end{itemize}
}

Se recomienda \textbf{evitar} escribir texto dentro de la fecha:
{\color{Red}
\begin{itemize}
	\item \texttt{1992-1993 aprox.}
	\item \texttt{Siglo XX}
\end{itemize}
}

\subsection{Fecha de ingreso}
Para consignar la fecha de ingreso del material a la colección se recomienda usar el formato \textit{Little-endian}\footnote{Date format by country. \url{https://en.wikipedia.org/wiki/Date_format_by_country}} con la siguiente sintáxis \verb|día/mes/año|. El año \textbf{no} debe estar abreviado en formato de dos cifras. También es válido anotar solamente el año como fecha de ingreso. 

Ejemplos de formato recomendado:
{\color{Blue}
\begin{itemize}
	\item \texttt{31/10/2012}
	\item \texttt{1999}
	\item \texttt{2/9/2000}
\end{itemize}
}

Se recomienda \textbf{evitar} la representación \textit{Middle-endian} o notación norteamericana:
{\color{Red}
\begin{itemize}
	\item \texttt{10/30/2010}
	\item \texttt{15/5/99 (Año abreviado)}
\end{itemize}
}

\subsection{Fecha de descripción}
Consigna la fecha en que se ha elaborado la ficha de la unidad. Se recomienda que la descripción sea en el formato \textit{Little-endian}. Debido a que una ficha o registro puede ser modificado posteriormente, solamente se consigna la última fecha de modificación.

Ejemplos de formato recomendado:
{\color{Blue}
\begin{itemize}
	\item \texttt{24/06/2015}
	\item \texttt{01/05/2010}
	\item \texttt{20/5/1999}
\end{itemize}
}

Se recomienda \textbf{evitar} escribir texto dentro de la fecha o agregar varias fechas:
{\color{Red}
\begin{itemize}
	\item \texttt{22/9/2012, 24/9/2012}
	\item \texttt{Última modificación: 20/03/2012}
\end{itemize}
}

\section{Campos con múltiples nombres}
Los siguientes rubros consignan principalmente el nombre de la persona encargada de cierta tarea: \textbf{Investigación, Realización, Dirección, Guión, Adaptación, Idea original, Fotografía, Fotografía fija, Edición, Musicalización, Voces, Actores, Animación, Otros colaboradores, Entidad productora, Productor, Distribuidora, Datos del archivero}. 

En ocasiones es necesario enlistar varias personas o entidades, así que se recomienda separar cada persona o entidad por medio del símbolo de coma (\verb|,|). En ocasiones se requiere especificar la labor de una persona o entidad, por lo que se recomienda indicar la labor específica seguida del símbolo dos puntos (\verb|:|), seguido del nombre. Si varias personas comparten una labor específica se recomienda separar también con una coma y para diferenciar de otras labores o personas separar con el símbolo de punto y coma (\verb|;|). 
%Si solamente se indicará la especialización de una persona, se puede omitir el uso de punto y coma (\verb|;|) y solamnte separar con comas.

Ejemplos de formato recomendado:
{\color{Blue}
\begin{itemize}
	\item \texttt{Alain Resnais, Robert Hessens}
	\item \texttt{Director de fotografía: André A. Dantan; Asistente de fotografía: Marcel Fradetal}
	\item \texttt{Asistente: Jacqueline Grigaut-Lfévne, Damouré Zika; Direccion artística: René Bazar, Jean Rouch}
	\item \texttt{Georges Strouvé; Asistente de cámara: Patricio Guzmán}
	%\item \texttt{Asistente de cámara: Patricio Guzmán; Georges Strouvé}
	\item \texttt{Instituto Cubano del Arte, Industria Cinematográficas}
\end{itemize}
}

Se recomienda \textbf{evitar} el uso de otros caracteres o preposiciones para separar o delimitar una lista de personas:

{\color{Red}
\begin{itemize}
	\item \texttt{Alain Resnais y Robert Hessens}
	\item \texttt{Director de fotografía: André A. Dantan / Asistente de fotografía: Marcel Fradetal}
	\item \texttt{Instituto Cubano del Arte e Industria Cinematográficas}
\end{itemize}
}

\section{Duración}
El formato estricto es emplear horas, minutos y segundos separados por dos puntos (\verb|:|), sin embargo también es común encontrar el formato definido únicamente con minutos y segundos, en este caso se anota apóstrofe (\verb|'|) para indicar minutos y dos apóstrofes (\verb|''|) para indicar segundos.

Ejemplos de formato recomendado:
{\color{Blue}
\begin{itemize}
	\item \texttt{01:43:00}
	\item \texttt{0:10:6}
	\item \texttt{21'17''}
	\item \texttt{47''}
\end{itemize}
}

Se recomienda \textbf{evitar} el uso de otros caracteres para indicar minutos y segundos en notación abreviada. En caso de notación con dos puntos (\verb|:|) no omitir las horas (aunque sean con valor cero) para evitar ambigüedad:
{\color{Red}
\begin{itemize}
	\item \texttt{01:02 (Lo deseable sería 00:01:02)}
	\item \texttt{12\'{}40\'{}\'{} (Se utiliza acento en lugar de apóstrofe)}
	\item \texttt{90'10" (Se utilizan comillas en lugar de dos apóstrofes)}
\end{itemize}
}

\section{Fuentes y Recursos}
Ambos rubros registran la estructura del material que se está catalogando y ya existe una lista bien definida para catalogarlos, de manera que se recomienda seguir esta convención y en caso de incluir varios elementos para cada rubro, separar mediante comas (\verb|,|).

Ejemplos de formato recomendado:
{\color{Blue}
\begin{itemize}
	\item \texttt{Documental, registros fílmicos, testimonios orales}
	\item \texttt{Grabación de campo, voz en off, conducción, entrevistas}
\end{itemize}
}

Se recomienda \textbf{evitar} el uso de otro caracter o letra(s) para delimitar varios recursos o fuentes. Tener especial cuidado en no mezclar un recurso en una fuente y viceversa.
{\color{Red}
\begin{itemize}
	\item \texttt{Grabación de campo y música en off}
	\item \texttt{Grabación de campo, voz en off, música de época (\textit{Música de época} es una fuente al registrar recursos)}
	\item \texttt{Intertítulos, entrevistas (El elemento \textit{intertítulos} no pertenece a fuentes o recursos)}
\end{itemize}
}

Cabe hacer una reflexión en cuanto a cuáles elementos son considerados fuentes y recursos, ya que en ocasiones se registran elementos que no estan considerados (como el caso de \textit{intertítulos}) y en ocasiones se especifican dichos elementos (por ejemplo: música en off, fotografía aérea, pintura heráldica). Se recomienda determinar si es posible hacer está especificación, ya que es información valiosa que debe ser tratada de manera adecuada a través de un sistema automatizado.


\section{Campos Vacios}
En algúnos rubros que no se tiene información o que se encuentran vacíos, se tienen que indicar con palabras clave que no existe tal información, como lo son:
\begin{itemize}
	\item \texttt{Título atribuido (Area de Identificación): "\textbf{No Identificado}"}
	\item \texttt{Número de programa dentro de la serie o compilación (Área de Identificación): "\textbf{NN (No numerado)}"}
\end{itemize}

El problema es que hay ciertos rubros que están sin dicha información y no se ocupan las palabras claves para indicarlos por lo que se recomienda que mejor permanezcan vacios para un mejor lectura y eficiencia de espacio en la base de datos.
De todas maneras el administrador puede ver qué campos estan vacíos en la página web, es decir, es posible asignar estos valores al notar que el campo correspondiente está vacio.

\section{Forma de Ingreso}
En este rubro se tienen tres opciones para describir cómo es que llega la unidad documental al acervo, pero esas opciones están restringidas a:

\begin{itemize}
	\item \texttt{Compra}
	\item \texttt{Donación}
	\item \texttt{Produción propia}
\end{itemize}

El problema es que en algunos casos se agregó de quién fue la donación o compra, entonces se tiene la duda de que si es necesario e importante guardar dicha información.

\end{document}
