% Preamble
\documentclass[10pt,letterpaper]{article}
\usepackage[spanish]{babel}

% Packages
\selectlanguage{spanish}
\usepackage[utf8]{inputenc} % For spanish (and international) letters like acents.
\usepackage{hyperref} % To create hyperlinks within the document.
\usepackage{graphicx} % To include graphics (pictures, images)
%\usepackage{float} % For the use of the parameter "H" in command "\begin{figure}[H]" (i.e. exact position of image in text)
\usepackage{verbatim} % For long comments
%\usepackage{color} % For color in text, background text and page background.
\usepackage[usenames,dvipsnames]{xcolor} % For color in text, background text and page background.

% Document environment
\begin{document}

% Top matter
\title{Recomendaciones para el Manual de Catalogación del Acervo de Documental en Video}
\author{Rodrigo Eduardo Colín Rivera}
\date{\today}
\maketitle

\setcounter{secnumdepth}{0} % Non-numbering sections
\setcounter{tocdepth}{0} % Non-numnbering table of contents
%\graphicspath{{../Diagramas/}} % Path of the folder containing the images

\section{Antecedentes}
El Manual de Catalogación del Acervo de Documental en Video para la Serie de Documentales en Video del Laboratorio Audiovisual de Investigación Social del Instituto Mora es un manual que expone las características de los diversos rubros que integra la ficha de documentación gestada con base en la adaptación de la norma ISAD(G) (\textit{General International Standard Archival Description}).

Durante el desarrollo del sistema de apoyo a la catalogación de archivos audiovisuales se tomó la decisión de crear una \textbf{base de datos} que almacene cada rubro estipulado en dicho manual. Sin embargo, dado que la información debe ser comprensible tanto para personas como para la máquina, es decir, se requieren algunos convenciones para que la información pueda ser procesada de manera automática sin perder la expresividad descrita por las personas o catalogadores.

A continuación se describen los formatos deseables de algunos campos que facilitarán el proceso de automatizar y manejar la información del ccervo de documental en video.

\section{Fechas}

\subsection{Fecha}
Comúnmente este rubro consigna el \textbf{año} en el que se terminó la producción de la obra. La fecha puede comprender un periodo de tiempo determinado indicando el año de inicio y el año de finalización de la obra y en caso de desconocer este dato, se aproxima la fecha entre corchetes. 

Ejemplos de formato recomendado:
{\color{Blue}
\begin{verbatim}
1956
1948-1949
[1982]
[1992-1993]
\end{verbatim}
}

Se recomienda \textbf{evitar} escribir texto dentro de la fecha:
{\color{Red}
\begin{verbatim}
1992-1993 aprox. 
Siglo XX
\end{verbatim}
}

\subsection{Fecha de ingreso}
Para consignar la fecha de ingreso del material a la colección se recomienda usar el formato \textit{Little-endian}\footnote{Date format by country. \url{https://en.wikipedia.org/wiki/Date_format_by_country}} con la siguiente sintáxis \verb|día/mes/año|. El año \textbf{no} debe estar abreviado en formato de dos cifras. También es válido anotar solamente el año como fecha de ingreso. 

Ejemplos de formato recomendado:
{\color{Blue}
\begin{verbatim}
31/10/2012
1999
2/9/2000
\end{verbatim}
}

Se recomienda \textbf{evitar} la representación \textit{Middle-endian} o notación norteamericana:
{\color{Red}
\begin{verbatim}
10/30/2010
15/5/99 (Año abreviado)
\end{verbatim}
}

\subsection{Fecha de descripción}
Consigna la fecha en que se ha elaborado la ficha de la unidad. Se recomienda que la descripción sea en el formato \textit{Little-endian}. Debido a que una ficha o registro puede ser modificado posteriormente, solamente se consigna la última fecha de modificación.

Ejemplos de formato recomendado:
{\color{Blue}
\begin{verbatim}
24/06/2015
01/05/2010
20/5/1999
\end{verbatim}
}

Se recomienda \textbf{evitar} escribir texto dentro de la fecha o agregar varias fechas:
{\color{Red}
\begin{verbatim}
22/9/2012, 24/9/2012
Última modificación: 20/03/2012
\end{verbatim}
}

\section{Campos con múltiples nombres}
Los siguientes rubros consignan principalmente el nombre de la persona encargada de cierta tarea: \textbf{Investigación, Realización, Dirección, Guión, Adaptación, Idea original, Fotografía, Fotografía fija, Edición, Musicalización, Voces, Actores, Animación, Otros colaboradores, Entidad productora, Productor, Distribuidora, Datos del archivero}. 

En ocasiones es necesario enlistar varias personas o entidades, así que se recomienda separar cada persona o entidad por medio del símbolo de coma (\verb|,|). En ocasiones se requiere especificar la labor de una persona o entidad, por lo que se recomienda indicar la labor específica seguida del símbolo dos puntos (\verb|:|), seguido del nombre. Si varias personas comparten una labor específica se recomienda separar también con una coma y para diferenciar de otras labores o personas separar con el símbolo de punto y coma (\verb|;|). 
%Si solamente se indicará la especialización de una persona, se puede omitir el uso de punto y coma (\verb|;|) y solamnte separar con comas.

Ejemplos de formato recomendado:
{\color{Blue}
\begin{verbatim}
Alain Resnais, Robert Hessens
Director de fotografía: André A. Dantan; Asistente de fotografía: Marcel Fradetal
Asistente: Jacqueline Grigaut-Lfévne, Damouré Zika; Direccion artística: René Bazar, Jean Rouch
Georges Strouvé; Asistente de cámara: Patricio Guzmán
Asistente de cámara: Patricio Guzmán; Georges Strouvé
Instituto Cubano del Arte, Industria Cinematográficas
\end{verbatim}
}

Se recomienda \textbf{evitar} el uso de otros caracteres o preposiciones para separar o delimitar una lista de personas:
{\color{Red}
\begin{verbatim}
Alain Resnais y Robert Hessens
Director de fotografía: André A. Dantan / Asistente de fotografía: Marcel Fradetal
Instituto Cubano del Arte e Industria Cinematográficas
\end{verbatim}
}

\section{Duración}
El formato estricto es emplear horas, minutos y segundos separados por dos puntos (\verb|:|), sin embargo también es común encontrar el formato definido únicamente con minutos y segundos, en este caso se anota apóstrofe (\verb|'|) para indicar minutos y dos apóstrofes (\verb|''|) para indicar segundos.

Ejemplos de formato recomendado:
{\color{Blue}
\begin{verbatim}
01:43:00
0:10:6
21'17''
47''
\end{verbatim}
}

\end{document}